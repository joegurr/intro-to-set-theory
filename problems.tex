\documentclass{article}
\usepackage[utf8]{inputenc}
\usepackage{enumitem}
\usepackage{amsmath}
\usepackage{amssymb}
\usepackage{hyperref}
% \usepackage[margin=1in]{geometry}
\hypersetup{
    colorlinks=true,
    linkcolor=blue,
    urlcolor=blue,
}

\title{Introduction to Set Theory Tutorial}
\author{Joe Gurr}
\date{\today}

\begin{document}

\maketitle

Welcome to my first tutorial session on Set Theory! My name is Joe.
I want this to be an open, judgement-free space to ask questions and learn.
I'd really appreciate any questions you have at any stage, so  please, please feel free to interrupt me!

I'd like to begin by saying Set Theory as a separate mathematical discipline is a relatively
modern branch of mathematics.
It is also quite abstract, and it could take some time for it to sink in fully.
It took me some time to appreciate why it exists and its inherent beauty,
hopefully I can help make it easier for you to understand.

I know that this will be the first time many of you will have been exposed
to Set Theory, so this tutorial sheet should be approached in conjunction with a brief slideshow
I have prepared going through the very fundamentals of Set Theory.
If you haven't yet got a copy of this please reach out!

I'd like to group people together and give you some time to work things out.
Just remember that I'm here at any stage if you need a hand at all!

\newpage

\begin{enumerate}[label=(\alph*)]
    \item Are the following statements True or False:
          \begin{enumerate}[label=(\roman*)]
              \item For all sets $X$, $\emptyset \subseteq X$ (reminder: check for $X = \emptyset$)
              \item $|\emptyset| = 0$
              \item $|\{\emptyset\}| = 0$
              \item If $A \subseteq B$ and $B \subseteq A$ then $A = B$
              \item There exits a set $X$ such that $|X| < 0$
          \end{enumerate}
    \item We define the \underline{Cartesian Product} between two sets
          $A$ and $B$ as
          \begin{equation*}
              A \times B = \{(x,y) \text{ where } x \in A \text{ and } y \in B\}
          \end{equation*}
          \begin{enumerate}[label=(\roman*)]
              \item What is $A \times B$ where $A = \{1, 2, 3\}$ and $B = \{10, 11\}$?
              \item What is $C \times D$ where $C = \{ \{1, 2\}, 3\}$ and $B = \{\emptyset, \{\emptyset\}\}$?
          \end{enumerate}
    \item What does $\mathbb{R} \times \mathbb{R}$ represent? Where have you seen this set before?

          (\textit{note}: $\mathbb{R} \times \mathbb{R}$ is also often called $\mathbb{R}^2$)
    \item Express $|A \times B|$ in terms of $|A|$ and $|B|$
    \item We define the \underline{Set Difference} between two sets $A$ and $B$ as
          \begin{equation*}
              A - B = \{ x \text{ where } x \in A \text{ and } x \notin B \}
          \end{equation*}
          Calculate the following Set Differences:
          \begin{enumerate}[label=(\roman*)]
              \item Let $A =\{ 1, 2, 3\}$ and $B$ be the set of even numbers. What is $A - B$
              \item What is $\mathbb{Z} - \mathbb{N}$
          \end{enumerate}
          (\textit{note}: Set Difference is also commonly denoted by $A \setminus B$)
    \item We define the \underline{Complement} of a set $A$ with respect to a particular superset $B$ (recall that a superset means that $A \subseteq B$)
          \begin{equation*}
              A^{'} = \{ x \text{ where } x \in B \text{ and } x \notin A \}
          \end{equation*}
          \begin{enumerate}[label=(\roman*)]
              \item Let $A = \{1, 2, 3\}$ and $B = {1, 2, 3, 4, 5}$, what is the complement of $A$ with respect to $B$?
              \item What is the complement of $\mathbb{Q}$ with respect to the superset $\mathbb{R}$?
          \end{enumerate}
          (\textit{note}: the Complement of a set is also commonly denoted $A^c$)

          (\textit{question}: are Set Difference and Complement the same? If not, how do they differ?)
    \item \textbf{(Bonus)} Research and provide a definition of how "relations" and "functions" are defined in Set Theory. What is the difference between a relation and a function?
\end{enumerate}
\end{document}
