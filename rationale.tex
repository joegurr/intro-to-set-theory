\documentclass{article}
\usepackage[margin=1.3in]{geometry}
\begin{document}
\title{How My Five Minute Mock Tutorial Would Differ From An Hour Tutorial}
\author{Joe Gurr}
\date{\today}

\maketitle

I wanted to write this document to explain the differences between how I would run a ``regular'' tutorial
and how I will run the five minute mock tutorial. When I say ``regular'' tutorial I mean
``running this same tutorial in one hour instead of five minutes''.

If I were running a regular tutorial under the same premise,
where the students haven't been exposed to the concepts in a lecture beforehand,
I would use the slides and problem sheet I have prepared.
If the students have seen the material before
perhaps the slides aren't needed.

I see the main part of my role as ensuring that the classroom is a healthy and respectful space to
enable students to explore the problems.
I have deep faith in the students' ability to solve the problems when they are working together.
Of course I'll be there to provide assistance, guidance and feedback when appropriate.
(If you are curious as to what I mean by appropriate, please check out George Polya's excellent book on mathematics pedagogy
- \textbf{How to Solve it}! I'd love to talk to you about it!)

In a regular tutorial we'd spend some time initially talking about the concepts and the problems,
generally we'd then tackle some of the first problems as a class. That is intended with the problem sheet I have created!
Then I would divide the class into appropriate groups and set them a problem or two to work on together on whiteboards (or whatever big space is available).

Once most of the groups have solved their problems I would like to get each group to briefly speak
about the problem(s) they have solved.
This last step the depends greatly on the personalities and natures of the students,
I certainly wouldn't force anyone to do something they don't feel comfortable doing.

If I were running this particular tutorial in real life, in say, an hour's time, slot I would break it down as follows.
I would spend the first 20 minutes or so on the slides, then explain the problems and maybe solve one or two with the class.
Then I'd split the class up which would likely take another 10 minutes or so for things to settle and groups to start working.
Then I'd give the students about 20 minutes to solve the problems assigned and spend the last 10 minutes
going around the room talking about what each group has learned.

I have chosen the problems very intentionally to both reinforce what the students will have learned from the
slideshow, and to introduce them to new concepts that build off the previous ones.

In conclusion, my teaching style could best be described as collaborative
student-led problem solving!
I love to see groups of students at a whiteboard talking about the problems!

\end{document}