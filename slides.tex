\documentclass{beamer}
\usepackage{csquotes}
\usetheme{default}
\begin{document}
\title{Introduction to Set Theory}
\author{Joe Gurr}
\begin{frame}
    \titlepage
\end{frame}

\begin{frame}
    \tableofcontents
    \begin{displayquote}
        Every mathematician agrees that every mathematician must know some set theory;
        the disagreement begins in trying to decide how much is some - Paul Halmos
    \end{displayquote}
\end{frame}

\section{What is a set}

\begin{frame}{What is a set}
    A set is a collection of ``things''. \\~\

    What ``things'' you may ask? Almost anything you'd like ... \\~\
\end{frame}

\begin{frame}
    A collection of three colours, red, blue, and yellow is a set. So is the collection of the names of capital cities of Australia (or the world for that matter). \\~\

    We write a set by enclosing the collection of ``things'' inside $\{\text{brackets}\}$ with commas separating the different objects. \\~\

    \textcolor{blue}{e.g.} $\{\text{North}, \text{East}, \text{South}, \text{West}\}$ is a set \\~\

    Often we consider sets that contain numbers. Other times we consider sets that contain sets! \\~\

    \textcolor{blue}{e.g.} $\{17, 42, 108\}$ is also a set \\~\
\end{frame}

\begin{frame}
    Sets do not care about multiples of the same ``thing'', nor do they care about order \\~\

    \textcolor{blue}{e.g.} $\{\text{cat}, \text{dog}\}$ is the same as $\{\text{cat}, \text{dog}, \text{cat}\}$ \\~\

    \textcolor{blue}{e.g.} $\{1, 2, 3\}$ is the same as $\{3,2,1\}$
\end{frame}

\begin{frame}
    We use the symbol $\in$ to represent ``membership'' to a set. \\~\

    We write $1 \in \{1,2,3\}$ to say, 1 is a ``member of'' (or ``belongs to'')
    the set $\{1, 2, 3\}$. $4 \notin \{1,2,3\}$. \\~\

    We usually denote a set with a capital letter (usually starting at the start of the alphabet). \\~\

    We generally denote members of the set with a lower case letter (usually from the end of the alphabet). \\~\

    For example you will often see something like. Let $x$ be a member of the set $A$.
\end{frame}

\begin{frame}
    If all the members in one set, say set $A$, also belong to another set, say set $B$,
    we say that $A$ is a subset of $B$. We write this as $A \subseteq B$. \\~\

    $B$ may very well contain more ``things'' than $A$. \\~\

    For example the set $\{1,2,3\} \subseteq \{\cdots, -2, -1,0,1,2,\cdots\}.$
\end{frame}

\section{Some famous sets}

\begin{frame}{Some famous sets}
    \begin{itemize}
        \item The counting numbers $\{0, 1, 2, 3, 4, 5, \cdots\}$,
              otherwise known as the Natural Numbers, denoted by $\mathbb{N}$
        \item The Integers $\{\cdots, -2, -1, 0, 1, 2, \cdots\}$ denoted by $\mathbb{Z}$
        \item The Rational Numbers, i.e. those numbers that can be expressed as a ratio of two integers, denoted by $\mathbb{Q}$
        \item The Real Numbers, denoted by $\mathbb{R}$
        \item The set with no members $\{\}$, otherwise known as the Empty Set, denoted by $\emptyset$
    \end{itemize}
\end{frame}

\section{What can we do with sets}

\begin{frame}{What can we do with sets}
    Create a new set by combining the members of other sets. We call this the \underline{Union} of sets.
    We write the union of two sets $A$ and $B$ as $A \cup B$. \\~\

    \textcolor{blue}{e.g.} Let $A = \{a, b, c\}$ and let $B = \{x, y\}$, then $A \cup B = \{a, b, c, x, y\}$
\end{frame}

\begin{frame}{What can we do with sets}
    Create a new set by including only the common members of other sets. We call this the \underline{Intersection} of sets.
    We write the intersection of two sets $C$ and $D$ as $C \cap D$. \\~\

    \textcolor{blue}{e.g.} Let $C = \{-2, 0, 2\}$ and let $D = \{-1, 0, 1\}$, then $C \cap D = \{0\}$
\end{frame}

\begin{frame}{What can we do with sets}
    Count them ... well, kind of, ... We call this the \underline{Cardinality} of a set.
    Formally it is not really a ``count'', but we will get to that another day!
    The number of elements (the Cardinality) of a set $A$ is is denoted by $|A|$ \\~\

    \textcolor{blue}{e.g.} Let $A = \{a, b, c\}$, then $|A| = 3$.
\end{frame}

\begin{frame}
    We often write a set explicitly, by listing out all of the elements in the set. \\~\

    This doesn't work when we want to write out infinite sets. \\~\

    In these cases we use a ``generating rule'' (or rules) to define how a set is generated. \\~\

    For example, we could define the even integers the following way: $\{2x \text{ such that } x \in \mathbb{Z}\}$
\end{frame}

\section{Big results in set theory}

\begin{frame}{Big results in set theory}
    I'd like to end this session by briefly bringing up some of the most famous results of Set Theory.
    I want to give the beginning of an answer to the question ``Why do we care about sets?'' \\~\

    I don't expect anyone to completely understand any of this,
    but I think it is important to at least name of some of the most significant results.
    So that anyone interested can look it up themselves (or talk to me!). \\~\
    \begin{itemize}
        \item Formal ways of building the real numbers (via Dedekind cuts for example)
        \item There are more real numbers than natural numbers (Cantor's Diagonalisation Argument)
        \item Mathematics doesn't play as nice as we would have liked (Godel's Incompleteness Theorems)
    \end{itemize}
\end{frame}

\section{Further reading}

\begin{frame}{Further reading}
    I'd recommend the following two resources! \\~\
    \begin{itemize}
        \item \href{https://plato.stanford.edu/entries/set-theory/}{Stanford Encyclopedia of Philosophy - Set Theory}
        \item \href{https://en.wikipedia.org/wiki/Naive_Set_Theory_(book)}{Naive Set Theory by Paul Halmos}
    \end{itemize}
\end{frame}

\end{document}